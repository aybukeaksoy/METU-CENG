\documentclass[10pt,a4paper, margin=1in]{article}
\usepackage{fullpage}
\usepackage{amsfonts, amsmath, pifont}
\usepackage{amsthm}
\usepackage{graphicx}
\usepackage{float}

\usepackage{tkz-euclide}
\usepackage{tikz}
\usepackage{pgfplots}
\pgfplotsset{compat=1.13}

\usepackage{geometry}
 \geometry{
 a4paper,
 total={210mm,297mm},
 left=10mm,
 right=10mm,
 top=10mm,
 bottom=10mm,
 }
 % Write both of your names here. Fill exxxxxxx with your ceng mail address.
 \author{
  Aksoy, Aybüke\\
  \texttt{e2448090@ceng.metu.edu.tr}
  \and
  Varlı, Yiğit\\
  \texttt{e2381036@ceng.metu.edu.tr}
}

\title{CENG 384 - Signals and Systems for Computer Engineers \\
Spring 2022 \\
Homework 2}
\begin{document}
\maketitle



\noindent\rule{19cm}{1.2pt}

\begin{enumerate}

\item %write the solution of q1
    \begin{enumerate}
    % Write your solutions in the following items.
    \item Following the graph, we can find out that:
    \[x(t)-\frac{2dx(t)}{dt}+ 3y(t)-2 \int_{-\infty}^{t} y(t)dt=\frac{dy(t)}{dt} \]
    \item To ease our job, we can differentiate both sides and get rid of the integral. Then, we get the following equation:
    \[\frac{dx(t)}{dt}-\frac{2d^2x(t)}{dt^2}+ 3\frac{dy(t)}{dt}-2y(t)=\frac{d^2y(t)}{dt^2} \]
    \[\frac{2d^2y(t)}{dt^2}- 3\frac{dy(t)}{dt}+2y(t)=-\frac{2d^2x(t)}{dt^2}+\frac{dx(t)}{dt}\]
    Now, we can start solving by using the linearity of the system:
    \[y_g(t)=y_h(t)+y_p(t)\]
     \[x(t)=x_1(t)+x_2(t) \ where \ x_1(t)=e^{-t}u(t) \ and \ x_2(t)=e^{-2t}u(t)\] 
     1) Finding the partial solution:\\
     We know that $y_p(t) \ and \ x(t)$ is in the form:
      \[y_p(t)=H(\lambda)e^{\lambda t} \ , \ x(t)=e^{\lambda t}\] 
    Taking the derivatives:
     \[y_p'(t)=\lambda H(\lambda)e^{\lambda t} \ , \ y_p''(t)=\lambda^{2}H(\lambda)e^{\lambda t} \ , \ 
     x'=\lambda e^{\lambda t} \ , \ x''=\lambda^{2} e^{\lambda t}
     \] 
     Putting them into the equation:
     \[\lambda^{2} H(\lambda)e^{\lambda t}-3\lambda H(\lambda)e^{\lambda t}+2 H(\lambda)e^{\lambda t}=\lambda e^{\lambda t}-2\lambda^{2} e^{\lambda t} \]
     \[H(\lambda)= \frac{\lambda-2\lambda^2}{\lambda^2-3\lambda +2} \ (transfer \  function)\]
     \[y_{p1}=H(-1)e^{-t}u(t)=-\frac{1}{2}e^{-t}u(t)\]
     \[y_{p2}=H(-2)e^{-2t}u(t)=-\frac{5}{6}e^{-2t}u(t)\]
     \[y_{p}=-\frac{1}{2}e^{-t}u(t) -\frac{5}{6}e^{-2t}u(t)\]
     2) Finding the homogeneous solution:\\\\
     We know that $y_h$ is in the form:
     \[y_h(t)=Ce^{\alpha t}\]
     Taking the derivatives:
     \[y_h'(t)=\alpha Ce^{\alpha t} \ , \ y_h''(t)=\alpha^2 Ce^{\alpha t}\]
     Putting it into the equation $\frac{2d^2y(t)}{dt}- 3\frac{dy(t)}{dt}+2y(t)=0$ we get;
     \[\alpha^2 Ce^{\alpha t}-3\alpha Ce^{\alpha t}+2Ce^{\alpha t} = 0\]
     \[Ce^{\alpha t}(\alpha^2-3\alpha+2)=0\]
     Since C is a nonzero constant and $e^{\alpha t}$ cannot be 0 for any t,
     \[\alpha^2-3\alpha+2=0\]
     \[\alpha_1=1 \ , \ \alpha_2=2\]
    \[y_h(t)=C_1e^{t} + C_2e^{2t} \]
     \[y_g(t)=C_1e^{t} + C_2e^{2t}  -\frac{1}{2}e^{-t}u(t) -\frac{5}{6}e^{-2t}u(t)\]
     Since the system is initially at rest, y(0)=0 and y'(0)=0:\\
     \[y_g(0)=C_1 + C_2  -\frac{1}{2} -\frac{5}{6}=0 \ ,  \ C_1+C_2= \frac{4}{3}\]
     \[y'_g(0)=C_1 + 2C_2  +\frac{1}{2} +\frac{10}{6}=0 \ ,  \ C_1+2C_2= -\frac{13}{6}\]
     \[C_1=\frac{29}{6} \ , \ C_2=-\frac{7}{2}\]
     \[y_g(t)=\frac{29}{6}e^{t} - \frac{7}{2}e^{2t}  -\frac{1}{2}e^{-t}u(t) -\frac{5}{6}e^{-2t}u(t)\]
    \end{enumerate}

\item %write the solution of q2
    \begin{enumerate}
    % Write your solutions in the following items.
    \item We know that $x[n] \ast \delta[n-t] = x[n-t]$. By using this convolution result and distributive property of convolution:
    \[y[n] = x[n] \ast h[n]\]
    \[y[n] = (\delta[n-1] + 3\delta[n+2]) \ast (2\delta[n+2] - \delta[n+1])\]
    \[y[n] = (\delta[n-1] \ast 2\delta[n+2]) - (\delta[n-1] \ast \delta[n+1]) + (3\delta[n+2] \ast 2\delta[n+2]) - (3\delta[n+2] \ast \delta[n+1])\]
    \[y[n] = 2\delta[n+1] - \delta[n] + 6\delta[n+4] - 3\delta[n+3]\]
    
    \begin{filecontents}{fig1.dat}
    n   xn
    -1 2
    0 -1
    -4  6
    -3  -3
    \end{filecontents}

     \begin{figure}[H]
     \centering
     \begin{tikzpicture}[scale=1.0] 
       \begin{axis}[
           axis lines=middle,
           xlabel={$n$},
           ylabel={$y[n]$},
           xtick={-5,-4, ...,4},
           ytick={-3,-2, ...,7},
           ymin=-3, ymax=7,
           xmin=-5, xmax=4,
           every axis x label/.style={at={(ticklabel* cs:1.05)}, anchor=west,},
           every axis y label/.style={at={(ticklabel* cs:1.05)}, anchor=south,},
           grid,
         ]
         \addplot [ycomb, black, thick, mark=*] table [x={n}, y={xn}] {fig1.dat};
       \end{axis}
     \end{tikzpicture}
     \caption{$n$ vs. $y[n]$.}
     \label{fig:fig1}
     \end{figure}
    
    \item By using the property of $u[n] - u[n-1] = \delta[n]$, we can write $x[n]$ and $h[n]$ as in the following way:
    \[x[n] = \delta[n+1] + \delta[n] + \delta[n-1]\]
    \[h[n] = \delta[n-4] + \delta[n-5]\]
    By using the same properties in the part a:
    \[y[n] = (\delta[n+1] + \delta[n] + \delta[n-1]) \ast (\delta[n-4] + \delta[n-5])\]
    \[y[n] = (\delta[n+1] \ast \delta[n-4]) + (\delta[n+1] \ast \delta[n-5]) + (\delta[n] \ast \delta[n-4]) + (\delta[n] \ast \delta[n-5]) + (\delta[n-1] \ast \delta[n-4]) + (\delta[n-1] \ast \delta[n-5])\]
    \[y[n] = \delta[n-3] + 2\delta[n-4] + 2\delta[n-5] + \delta[n-6]\]
    
        \begin{filecontents}{fig2.dat}
    n   xn
    3   1
    4   2
    5   2
    6   1
    \end{filecontents}

     \begin{figure}[H]
     \centering
     \begin{tikzpicture}[scale=1.0] 
       \begin{axis}[
           axis lines=middle,
           xlabel={$n$},
           ylabel={$y[n]$},
           xtick={-2,-1, ...,7},
           ytick={-2,-1, ...,3},
           ymin=-2, ymax=3,
           xmin=-2, xmax=7,
           every axis x label/.style={at={(ticklabel* cs:1.05)}, anchor=west,},
           every axis y label/.style={at={(ticklabel* cs:1.05)}, anchor=south,},
           grid,
         ]
         \addplot [ycomb, black, thick, mark=*] table [x={n}, y={xn}] {fig2.dat};
       \end{axis}
     \end{tikzpicture}
     \caption{$n$ vs. $y[n]$.}
     \label{fig:fig1}
     \end{figure}
    \end{enumerate}

\item %write the solution of q3  
    \begin{enumerate}
    % Write your solutions in the following items.
    \item %write the solution of q3a
        \begin{equation*}
h(t) = \left\{
        \begin{array}{ll}
            e^{-\frac{t}{2}} & \quad t \geq 0 \\
            0 & \quad t < 0
        \end{array}
    \right.
\end{equation*}
   \begin{equation*}
x(t) = \left\{
        \begin{array}{ll}
            e^{-t} & \quad t \geq 0 \\
            0 & \quad t < 0
        \end{array}
    \right.
\end{equation*}
\[y(t)=x(t) \ast h(t)= \int_{-\propto}^{\propto}x(\tau)h(t-\tau)d\tau\]
\[ x(\tau)=e^{-\tau}u(\tau) \ , \  h(t-\tau)=e^{-\frac{1}{2}(t-\tau)}u(t-\tau)\]
\[y(t)=\int_{-\propto}^{\propto}e^{-\tau}u(\tau)e^{-\frac{1}{2}(t-\tau)}u(t-\tau)d\tau\]
This integral takes nonzero values only when t$\geq \tau \geq 0$ since $u(\tau) \neq 0 \ when \  \tau \geq 0 \ and \ u(t-\tau) \neq 0 \ when \ t-\tau \geq 0$
Hence, 
\[y(t)=\int_{0}^{t}e^{-\tau}e^{-\frac{1}{2}(t-\tau)}d\tau\] 
\[y(t)=\int_{0}^{t}e^{-\tau}e^{\frac{1}{2}\tau}e^{-\frac{1}{2}t}d\tau\]
\[y(t)=\int_{0}^{t}e^{-\frac{1}{2}\tau}e^{-\frac{1}{2}t}d\tau\]
Since $e^{-\frac{1}{2}t}$ does not depend on $\tau$, it is a constant.
\[y(t)=e^{-\frac{1}{2}t}\int_{0}^{t}e^{-\frac{1}{2}\tau}d\tau\]
Taking the integral:
\[y(t)=e^{-\frac{1}{2}t}\Big[-2e^{-\frac{1}{2}\tau}\Big|_0^t \Big]\]
\[y(t)=(2-2e^{-\frac{1}{2}t})e^{-\frac{1}{2}t}\]
\[y(t)=2e^{-\frac{t}{2}}-2e^{-t}\]
As this only holds for the values $t \geq 0$
\[y(t)=(2e^{-\frac{t}{2}}-2e^{-t})u(t)\]
    \item %write the solution of q3b
     \begin{equation*}
h(t) = \left\{
        \begin{array}{ll}
            e^{-3t} & \quad t \geq 0 \\
            0 & \quad t < 0
        \end{array}
    \right.
\end{equation*}\\
   \begin{equation*}
x(t) = \left\{
        \begin{array}{ll}
            0 & \quad t > 4 \\
            1 & \quad 0 \leq t \leq 4\\
            0 & \quad t < 0 
        \end{array}
    \right.
\end{equation*}
\[y(t)=x(t) \ast h(t)= \int_{-\propto}^{\propto}x(\tau)h(t-\tau)d\tau\]
\[ x(\tau)=u(\tau)-u(\tau-4) \ , \  h(t-\tau)=e^{-3(t-\tau)}u(t-\tau)\]
\[y(t)=\int_{-\propto}^{\propto}(u(\tau)-u(\tau-4))e^{-3(t-\tau)}u(t-\tau)d\tau\]
\[y(t)=\int_{-\propto}^{\propto}(u(\tau)e^{-3(t-\tau)}u(t-\tau)d\tau \ - \ \int_{-\propto}^{\propto}u(\tau-4)e^{-3(t-\tau)}u(t-\tau)d\tau \]

First part of the integral takes nonzero values only for $t \geq \tau \geq 0$ and the second part of the integral takes nonzero values only for $t \geq \tau \geq 4$\\
Hence,
\[y(t)=\int_{0}^{t}e^{-3(t-\tau)}d\tau \ - \ \int_{4}^{t}e^{-3(t-\tau)}d\tau \]
Since $e^{-3t}$ does not depend on $\tau$, it is a constant. 
\[y(t)=e^{-3t}\int_{0}^{t}e^{3\tau}d\tau \ - \ e^{-3t}\int_{4}^{t}e^{3\tau}d\tau \]
Taking the integral:
\[y(t)=e^{-3t}\Big[\frac{1}{3}e^{3\tau}\Big|_0^t \Big] u(t) \ - \ e^{-3t}\Big[\frac{1}{3}e^{3\tau}\Big|_4^t \Big] u(t-4)\]
\[y(t)=e^{-3t}\Big[\frac{1}{3}e^{3t}-\frac{1}{3} \Big] u(t) - e^{-3t}\Big[\frac{1}{3}e^{3t} - \frac{1}{3}e^{12} \Big] u(t-4)\]
\[y(t)=(\frac{1}{3}-\frac{1}{3}e^{-3t})u(t) + (\frac{1}{3}e^{-3t + 12} -\frac{1}{3})u(t-4)\]
\end{enumerate}
    
\item %write the solution of q4
    \begin{enumerate}
    % Write your solutions in the following items.
    \item %write the solution of q4a
    We know that:
    \[y(t)=x(t) \ast h(t)= \int_{-\propto}^{\propto}x(\tau)h(t-\tau)d\tau\]
    \[x(t)=x(t) \ast \delta(t)= \int_{-\propto}^{\propto}x(\tau)\delta(t-\tau)d\tau\]
    and also, 
    \[x(t)=\delta(t) \ast x(t) = \int_{-\propto}^{\propto}x(t-\tau)\delta(\tau)d\tau\]
    Hence we can directly write $\delta(\tau)$ in place of $x(\tau)$ in the first equation to obtain h(t) and it would give us:
     \[\int_{-\propto}^{\propto}\delta(\tau)h(t-\tau)d\tau=\delta(t)\ast h(t)=h(t)\]
     y(t) is given as:
     \[y(t)=\int_{-\propto}^{t}e^{-(t-\tau)}x(\tau-3)d\tau\]
     in the question.
     We can write it as:
      \[h(t)=\int_{-\propto}^{t}e^{-(t-\tau)}\delta(\tau-3)d\tau\] 
      \[h(t)=e^{-(t-3)} \ for \ t \geq 3\] 
      \[h(t)=e^{-(t-3)}u(t-3)\] 
    \begin{center}
        \begin{figure}[H]
        \hspace{15mm}\includegraphics[scale=0.5]{image.png}
        \caption{Graph of h(t)}
        \label{fig:htgraph}
    \end{figure}
    \end{center}
    \item %write the solution of q4b
     \[y(t)=\int_{-\propto}^{\propto}x(\tau)h(t-\tau)d\tau = \int_{-\propto}^{\propto}x(t-\tau)h(\tau)d\tau\]
     \[x(t)=u(t+2)-u(t-1)\] 
     \[y(t)=\int_{-\propto}^{\propto}e^{-(t-\tau-3)}u(t-\tau-3)(u(\tau+2)-u(\tau-1))d\tau\]
     \[y(t)=\int_{-2}^{t-3}e^{-(t-\tau-3)}d\tau - \int_{1}^{t-3}e^{-(t-\tau-3)}d\tau\]
     \[y(t)=e^{3-t}\int_{-2}^{t-3}e^{\tau}d\tau - e^{3-t}\int_{1}^{t-3}e^{\tau}d\tau\]
     \[y(t)=e^{3-t}(e^{t-3}-e^{-2})u(t-1) - e^{3-t}(e^{t-3}-e)u(t-4)\]
     \[y(t)=(1-e^{1-t})u(t-1) - (1-e^{4-t})u(t-4)\]
     
    \end{enumerate}
    
\item %write the solution of q5
    \begin{enumerate}   
    % Write your solutions in the following items.
    \item We know that convolution of a function with inverse of itself is $\delta[n]$. By using this fact:
    \[h_1[n] \ast h_1^{-1} = \delta[n]\]
    \[h_1[n] \ast ((\frac{1}{2})^n u[n]) =  \delta[n]\]
    \[h_1[n] = \delta[n] - \frac{1}{2}\delta[n-1]\]
    We can now apply convolution operation to function itself:
    \[h_1[n] \ast h_1[n] = (\delta[n] - \frac{1}{2}\delta[n-1]) \ast (\delta[n] - \frac{1}{2}\delta[n-1])\]
    We can use distribution property of convolution as we used in the second question. After all the operations we can find:
    \[h_1[n] \ast h_1[n] = \delta[n] - \delta[n-1] + \frac{1}{4}\delta[n-2]\]
    \item We know that $h[n] = h_0[n] \ast h_1[n] \ast h_1[n]$. To find $h_0[n]$, we can convolute both side with $h_1^{-1}[n] \ast h_1^{-1}[n]$. Then, we get the following:
    \[h[n] \ast h_1^{-1}[n] \ast h_1^{-1}[n] = h_0[n]\]
    Using the commutative property of convolution, we can first calculate:
    \[h_1^{-1}[n] \ast h_1^{-1}[n] = (\frac{1}{2})^n u[n] \ast (\frac{1}{2})^n u[n]\]
    \[h_1^{-1}[n] \ast h_1^{-1}[n] = \sum_{k = -\infty}^{\infty} \frac{1}{2}^k u[k] \frac{1}{2}^{n-k} u[n-k]\]
    \[h_1^{-1}[n] \ast h_1^{-1}[n] = \sum_{k = -\infty}^{\infty} \frac{1}{2}^n u[k] u[n-k]\]
    \[h_1^{-1}[n] \ast h_1^{-1}[n] = \sum_{k = 0}^{n} \left(\frac{1}{2}\right)^n\]
    \[h_1^{-1}[n] \ast h_1^{-1}[n] = (n+1)\left(\frac{1}{2}\right)^n u[n]\]
    We know that $h_[n] = 4\delta[n] + \delta[n-2] -3\delta[n-3] + \delta[n-4]$ from the graph of $h[n]$. Now, we can convolute $h[n]$ with the result we found:
    \[h[n] \ast ((n+1)\left(\frac{1}{2}\right)^n u[n]) = h_0[n]\]
    \[\bigg(4\delta[n] + \delta[n-2] -3\delta[n-3] + \delta[n-4]\bigg) \ast \left((n+1)\left(\frac{1}{2}\right)^n u[n]\right)\]
    Again, using the distributive property of convolutions:
    \[h_0[n] = 4(n+1)\left(\frac{1}{2}\right)^n u[n] + (n-1)\left(\frac{1}{2}\right)^{n-2} u[n-2] - 3(n-2)\left(\frac{1}{2}\right)^{n-3} u[n-3] + (n-3)\left(\frac{1}{2}\right)^{n-4} u[n-4]\]
    \item From the block diagram, we can understand that $y[n] = x[n] \ast h_0[n]$. Thus:
    \[\hspace{-20mm}y[n] = \left(4(n+1)\left(\frac{1}{2}\right)^n u[n] + (n-1)\left(\frac{1}{2}\right)^{n-2} u[n-2] - 3(n-2)\left(\frac{1}{2}\right)^{n-3} u[n-3] + (n-3)\left(\frac{1}{2}\right)^{n-4} u[n-4]\right) \ast \Bigg(\delta[n] + \delta[n-2]\Bigg)\]
    We will divide convolution into $y_1[n] = h_0[n] \ast \delta[n]$ and $y_2[n] = h_0[n] \ast \delta[n-2]$:
    \[y[n] = y_1[n] + y_2[n]\]
    \[y_1[n] = 4(n+1)\left(\frac{1}{2}\right)^n u[n] + (n-1)\left(\frac{1}{2}\right)^{n-2} u[n-2] - 3(n-2)\left(\frac{1}{2}\right)^{n-3} u[n-3] + (n-3)\left(\frac{1}{2}\right)^{n-4} u[n-4]\]
    \[y_2[n] = 4(n-1)\left(\frac{1}{2}\right)^{n-2} u[n-2] + (n-3)\left(\frac{1}{2}\right)^{n-4} u[n-4] - 3(n-4)\left(\frac{1}{2}\right)^{n-5} u[n-5] + (n-5)\left(\frac{1}{2}\right)^{n-6} u[n-6]\]
    \end{enumerate}

\end{enumerate}

%Useful commands %%%%

% Function with cases
%$x(k) = \begin{cases} number1, & \mbox{if } k\mbox{ condition1} \\
%                                      number2 ,& \mbox{if } k\mbox{ condition} %\end{cases}$

% \begin{figure}[h!]
%     \centering
%         \begin{tikzpicture}[scale=1.0]
%           \begin{axis}[
%           axis lines=middle,
%           xlabel={$t$},
%           ylabel={$\boldsymbol{x(t)}$},
%           xtick={-4, -3, -2, -1, ..., 4},
%           ytick={-3, -2, -1, ..., 3},
%           ymin=-3, ymax=3,
%           xmin=-4, xmax=4,
%           every axis x label/.style={at={(ticklabel* cs:1.05)}, anchor=north,},
%           every axis y label/.style={at={(ticklabel* cs:1.05)}, anchor=south,},
%           grid,
%         ]
%           \path[draw,line width=4pt] (-4,0) -- (-3,0) -- (-2,0) -- (-1,1) -- (1,1) -- (1,0) -- (3,0) -- (4,0);
%           \end{axis}
%         \end{tikzpicture}
%         \caption{$t$ vs. $x(t)$.}
%         \label{fig:fig1}
%     \end{figure}
    
% \begin{filecontents}{fig2.dat}
%  n   xn
%  -1  1
%  0   0
%  1   0  
%  2   -2
%  3   0
%  4   3 
%  5   0
%  6   0
%  7   -4
% \end{filecontents}

% \begin{figure}[h!]
%     \centering
%     \begin{tikzpicture}[scale=1.0] 
%       \begin{axis}[
%           axis lines=middle,
%           xlabel={$n$},
%           ylabel={$\boldsymbol{x[n]}$},
%           xtick={ -1, 0,  ..., 7},
%           ytick={-4, -3, -2, -1, ..., 4},
%           ymin=-4, ymax=4,
%           xmin=-1, xmax=7,
%           every axis x label/.style={at={(ticklabel* cs:1.05)}, anchor=west,},
%           every axis y label/.style={at={(ticklabel* cs:1.05)}, anchor=south,},
%           grid,
%         ]
%         \addplot [ycomb, black, thick, mark=*] table [x={n}, y={xn}] {fig2.dat};
%       \end{axis}
%     \end{tikzpicture}
%     \caption{$n$ vs. $x[n]$.}
%     \label{fig:fig2}
% \end{figure}

%%%%%%%%%%%%%%%%%%%%

\end{document}

