\documentclass[10pt,a4paper, margin=1in]{article}
\usepackage{fullpage}
\usepackage{amsfonts, amsmath, pifont}
\usepackage{amsthm}
\usepackage{graphicx}
\usepackage{float}
\usepackage{amsmath}

\makeatletter
\newcommand\xleftrightarrow[2][]{%
  \ext@arrow 9999{\longleftrightarrowfill@}{#1}{#2}}
\newcommand\longleftrightarrowfill@{%
  \arrowfill@\leftarrow\relbar\rightarrow}
\makeatother

\usepackage{tkz-euclide}
\usepackage{tikz}
\usepackage{pgfplots}
\pgfplotsset{compat=1.13}

\usepackage{geometry}
 \geometry{
 a4paper,
 total={210mm,297mm},
 left=10mm,
 right=10mm,
 top=10mm,
 bottom=10mm,
 }
 % Write both of your names here. Fill exxxxxxx with your ceng mail address.
 \author{
  Aksoy, Aybüke\\
  \texttt{e2448090@ceng.metu.edu.tr}
  \and
  Varlı, Yiğit\\
  \texttt{e2381036@ceng.metu.edu.tr}
}

\title{CENG 384 - Signals and Systems for Computer Engineers \\
Spring 2022 \\
Homework 3}
\begin{document}
\maketitle



\noindent\rule{19cm}{1.2pt}

\begin{enumerate}

\item %write the solution of q1
    \begin{enumerate}
    % Write your solutions in the following items.
    \item %write the solution of q1a
    We know that;
    \[cos(wt)=\frac{1}{2}(e^{jwt}+e^{-jwt}) \ \ (a)\]
    \[sin(wt)=\frac{1}{2j}(e^{jwt}-e^{-jwt}) \ \ (b)\]
    So, we can write x(t) as;
    \[x(t)=\frac{1}{2}e^{j\frac{\pi}{4}t}+\frac{1}{2}e^{-j\frac{\pi}{4}t}+\frac{1}{2j}e^{j\frac{\pi}{5}t}-\frac{1}{2j}e^{-j\frac{\pi}{5}t} \ \ (1)\] 
    To be able to use the synthesis equation:
    \[x(t)=\Sigma_{k=-\infty}^{\infty}a_{k}e^{jk{w_0}t}\]
    We need to determine $w_0$.
    Since $w=\frac{2\pi}{T}$, T1=8 and T2=10.\\ 
    Therefore, the fundamental period $T_0$ is 40 and $w_0$ is $\frac{\pi}{20}.$
    \[x(t)=\Sigma_{k=-\infty}^{\infty}a_{k}e^{jk\frac{\pi}{20}t} \ \ (2)\] 
    Nonzero terms in the equation 2 should correspond to the terms in equation 1.\\
    Hence, the coefficient $a_k's$ from $-\infty$ to $\infty$ except $a_4, a_{-4}, a_5, a_{-5}$ are 0, and
    \[a_4=a_{-4}=\frac{1}{2j}\]
    \[a_5=a_{-5}=\frac{1}{2}\]
    \item %write the solution of q1b
    The spectral coefficients for discrete signals are periodic with the fundamental period $N_0$.\\
    To determine $N_0$ we need to find the period of each term in x[n]. 
    \[w=\frac{2\pi k}{N}\]
    Period for $sin(4\pi n)$, $N1=\frac{k}{2}$; period for $cos(2\pi n)$, N2=m and the period for $e^{j\pi n}$, N3=2.\\ 
    So, for k=4 and m=2, $N_0=2$\\
    As $N_0=2$ and the coefficients are periodic, to find $a_k's$ it is enough to calculate $a_0$ and $a_1$.\\
    Using the analysis equation with $N_0=2$:
    \[a_k=\frac{1}{2}\Sigma_{n=0}^{1}x[n]e^{-jk\pi n}\]
    \[x[0]=\frac{5}{2}\]
    \[x[1]=\frac{3}{2}+e^{j\pi}\]
    \[a_k=\frac{1}{2}(\frac{5}{2}+\frac{3}{2}e^{-jk\pi}+e^{j \pi(1-k)})\]
    Using the formula we obtained for $a_k$:
    \[a_0=\frac{3}{2} \ , \ a_1=1 \ and \ a_k=a_{k+2} \ \ for \ k>0\]
    \end{enumerate}

\item %write the solution of q2
    \[w_0=\frac{2\pi}{N}=\frac{2\pi}{7}\]
    Since we know the nonzero terms, by using the formula:
    \[x[n]=\Sigma_{k\subset N}a_ke^{-j{w_0}n}\]
    We can write x(t) as:
    \[x[n]=2je^{j\frac{2\pi}{7}n}-2je^{-j\frac{2\pi}{7}n}+2je^{j\frac{4\pi}{7}n}+2je^{-j\frac{2\pi}{7}n}+2je^{j\frac{6\pi}{7}n}-2je^{-j\frac{6\pi}{7}n}\]
    By using the equations a and b to write x(t) in the form:
    \[x(n)=4{j^2}sin(\frac{2\pi n}{7})+4cos(\frac{4\pi n}{7})+4{j^2}sin(\frac{6\pi n}{7})\]
    \[x(n)=-4sin(\frac{2\pi n}{7})+4cos(\frac{4\pi n}{7})-4sin(\frac{6\pi n}{7})\]
    \[x(n)=-4sin(\frac{2\pi n}{7})-4sin(\frac{4\pi n}{7}-\frac{\pi}{2})-4sin(\frac{6\pi n}{7})\]
    where $A_0$=0. 
\item %write the solution of q3  
    \begin{enumerate}
    % Write your solutions in the following items.
    \item We can write sin(wt) as the following form using Euler's Formula:
    \[sin(wt) = \frac{1}{2j}(e^{jwt} - e^{-jwt})\]
    Then, $w=\cfrac{\pi}{8}$
    \[sin(\frac{\pi}{8}t) = \frac{1}{2j}(e^{j\frac{\pi}{8}t} - e^{-j\frac{\pi}{8}t})\]
    \[sin(\frac{\pi}{8}t) = \frac{1}{2j}e^{j\frac{\pi}{8}t} - \frac{1}{2j}e^{-j\frac{\pi}{8}t}\]
    we know that $x(t)$ is of the form:
    \[x(t) = \sum_{k=-\infty}^{\infty}a_k e^{jkw_0t}\]
    When me make them equal, we found out that:
    \[a_{-1} = -\frac{1}{2j} \hspace{5mm} a_1 = \frac{1}{2j}\]
    Otherwise $a_k = 0$.\\
    %----------b------------
    \item We can write cos(wt) as the following form using Euler's Formula:
    \[cos(wt) = \frac{1}{2}(e^{jwt} + e^{-jwt})\]
    Then, $w=\cfrac{\pi}{8}$
    \[cos(\frac{\pi}{8}t) = \frac{1}{2}(e^{j\frac{\pi}{8}t} + e^{-j\frac{\pi}{8}t})\]
    \[cos(\frac{\pi}{8}t) = \frac{1}{2}e^{j\frac{\pi}{8}t} + \frac{1}{2}e^{-j\frac{\pi}{8}t}\]
    we know that $x(t)$ is of the form:
    \[x(t) = \sum_{k=-\infty}^{\infty}b_k e^{jkw_0t}\]
    When me make them equal, we found out that:
    \[b_{-1} = \frac{1}{2} \hspace{5mm} b_1 = \frac{1}{2}\]
    Otherwise $b_k = 0$.\\
    %-----------c--------------
    \item Given that $x(t) \xleftrightarrow{\text{FS}} a_k$ and $y(t) \xleftrightarrow{\text{FS}} b_k$ then,
    \[x(t)y(t) = z(t) \xleftrightarrow{\text{FS}} c_k = \sum_{l=-\infty}^{\infty} a_l b_{k-l}\]
    Since $a_l$ is nonzero only for $l=-1,1$, we can write the result of summation as:
    \[c_k = a_{-1}b_{k+1} + a_1b_{k-1}\]
    \[c_{-2} = a_{-1}b_{-1} + a_1b_{-3} = -\frac{1}{4j}\]
    \[c_{0} = a_{-1}b_1 + a_{1}b_{-1} = 0\]
    \[c_2 = a_{-1}b_3 + a_1b_1 = \frac{1}{4j}\]
    Otherwise $c_k = 0$.
    \end{enumerate}

\item From the given conditions, we can infer followings:
\begin{itemize}
    \item Since signal is odd, there is no DC component. Thus, $a_0=0$.
    \item For odd signals, we know that $a_k=-a_{-k}$. Thus, given that $a_2=3j$, $a_{-2}=-3j$.
    \item Given that 
    \[\frac{1}{4} \int_0^4 |x(t)|^2 dt = \sum_{k=-2}^{2} |a_k|^2 = 18\] 
    from the Parseval's Equality. We know that $|a_{-2}| = |a_2| = 3$ and $|a_0| = 0$. Using the Parseval's Equality, we find out that both $a_{-1}$ and $a_1$ is 0. Then, we know that:
    \[x(t) = \sum_{k=-\infty}^{\infty} a_k e^{jkw_0t}\]
    We are given that $T=4$ then we can infer $w_0$ from the equality of $w_0 = \frac{2\pi}{T} = \frac{\pi}{2}$. Using the $a_k$'s we can write signal as:
    \[x(t) = -3je^{-j2\frac{\pi}{2}t} + 3je^{j2\frac{\pi}{2}t}\]
    \[x(t) = 3j(e^{j\pi t} - e^{-j\pi t})\]
    Using the Euler's Formula:
    \[x(t) = -6sin(\pi t)\]
\end{itemize}


\item We can find the spectral coefficients from the equation:
    \[a_k = \frac{1}{N} \sum_{n=-\infty}^{\infty} f[n] e^{-jkw_0n}\]
    where $w_0 = \frac{2\pi}{N} = \frac{2\pi}{9}$ for both cases.
    \begin{enumerate}   
    % Write your solutions in the following items.
    \item 
    \[a_k = \frac{1}{9} \sum_{n=0}^{8} x[n]e^{-jk\frac{2\pi}{9}n}\]
    \[a_k = \frac{1}{9} \sum_{n=0}^{4}e^{-jk\frac{2\pi}{9}n}\]
    \[a_k = \frac{1}{9}(1 + e^{-jk\frac{2\pi}{9}} + e^{-jk\frac{4\pi}{9}} + e^{-jk\frac{6\pi}{9}} + e^{-jk\frac{8\pi}{9}})\]
    \item 
    \[b_k = \frac{1}{9} \sum_{n=0}^{8} y[n]e^{-jk\frac{2\pi}{9}n}\]
    \[b_k = \frac{1}{9} \sum_{n=0}^{3}e^{-jk\frac{2\pi}{9}n}\]
    \[b_k = \frac{1}{9}(1 + e^{-jk\frac{2\pi}{9}} + e^{-jk\frac{4\pi}{9}} + e^{-jk\frac{6\pi}{9}})\]
    \item Given that $x[n]$ and $H(e^{jw_0})$, we know that $y[n] = H(e^{jw_0}x[n]).$ Also, we know that relationship between the spectral coeffients of $y[n]$ and $x[n]$ is written as below:
    \[b_k = H(e^{jkw_0})a_k\]
    Then, we can write eigenvalue $H(e^{jkw_0})$ as:
    \[H(e^{jkw_0}) = \frac{b_k}{a_k}\]
    \[H(e^{jkw_0}) = \frac{1 + e^{-jkw_0} + e^{-2jkw_0} + e^{-3jkw_0}}{1 + e^{-jkw_0} + e^{-2jkw_0} + e^{-3jkw_0} + e^{-4jkw_0}}\]
    To get frequency response of the system from eigenvalue, we need to make $kw_0 \rightarrow w$ transformation. Then,
    \[H(e^{jw}) = \frac{1 + e^{-jkw} + e^{-2jkw} + e^{-3jkw}}{1 + e^{-jkw} + e^{-2jkw} + e^{-3jkw} + e^{-4jkw}}\]
    \end{enumerate}

 

\end{enumerate}


\end{document}

