\documentclass[10pt,a4paper, margin=1in]{article}
\usepackage{fullpage}
\usepackage{amsfonts, amsmath, pifont}
\usepackage{amsthm}
\usepackage{graphicx}
\usepackage{float}

\makeatletter
\newcommand\xleftrightarrow[2][]{%
  \ext@arrow 9999{\longleftrightarrowfill@}{#1}{#2}}
\newcommand\longleftrightarrowfill@{%
  \arrowfill@\leftarrow\relbar\rightarrow}
\makeatother

\usepackage{tkz-euclide}
\usepackage{tikz}
\usepackage{pgfplots}
\pgfplotsset{compat=1.13}

\usepackage{geometry}
 \geometry{
 a4paper,
 total={210mm,297mm},
 left=10mm,
 right=10mm,
 top=10mm,
 bottom=10mm,
 }
 % Write both of your names here. Fill exxxxxxx with your ceng mail address.
 \author{
  Varlı, Yiğit\\
  \texttt{e2381036@ceng.metu.edu.tr}
  \and
  Aksoy, Aybüke\\
  \texttt{e2448090@ceng.metu.edu.tr}
}

\title{CENG 384 - Signals and Systems for Computer Engineers \\
Spring 2022 \\
Homework 4}
\begin{document}
\maketitle



\noindent\rule{19cm}{1.2pt}

\begin{enumerate}

\item %write the solution of q1
    \begin{enumerate}
    % Write your solutions in the following items.
    \item %write the solution of q1a
    From the block diagram:
    \[\frac{dx(t)}{dt}-\int x(t)+x(t)-\int y(t)-2y(t)=\frac{dy(t)}{dt}\]
    Taking derivative with respect to t to get rid of the integrals:
    \[\frac{d^2x(t)}{dt^2}-x(t)+\frac{dx(t)}{dt}- y(t)-2\frac{dy(t)}{dt}=\frac{d^2y(t)}{dt^2}\]
    \[y^{''}+2y^{'}+y=x^{''}+x^{'}-x\]
    \item %write the solution of q1b
    \[(jw)^2Y(jw)+2jwY(jw)+Y(jw)=(jw)^2X(jw)+jwX(jw)-X(jw)\]
    
    \[Y(jw)((jw)^2+2jw+1)=X(jw)((jw)^2+jw-1)\]
    We know that:
    \[H(jw)=\frac{Y(jw)}{X(jw)}\]
    Hence, 
    \[H(jw)=\frac{(jw)^2+jw-1}{(jw)^2+2jw+1}=1+ \frac{-X-2}{(X+1)^2}\]
    To get the fractions:
    \[\frac{A}{X+1}+\frac{B}{(X+1)^2}=\frac{-X-2}{(X+1)^2}\]
    \[AX+A+B=-X-2\]
    \[A=-1 \ , \ B=-1\]
    So, 
    \[H(jw)=1-\frac{1}{jw+1}-\frac{1}{(jw+1)^2}\]
    \item %write the solution of q1c
    Taking the inverse fourier transform to find h(t) from H(jw) that we found in part b:
    \[H(jw)\xleftrightarrow[\text{}]{\text{F.T}}=h(t)\]
    \[1\xleftrightarrow[\text{}]{\text{F.T}}=\delta(t)\]
    \[\frac{1}{(jw+1)^2}\xleftrightarrow[\text{}]{\text{F.T}}=te^{-t}u(t)\]
    \[\frac{1}{jw+1}\xleftrightarrow[\text{}]{\text{F.T}}=e^{-t}u(t)\]
    \[h(t)=\delta(t)-(t+1)e^{-t}u(t)\]
    \item %write the solution of q1d
    We know that:
    \[Y(jw)=X(jw)H(jw)\]
    From the previous part:
    \[H(jw)=1-\frac{1}{jw+1}-\frac{1}{(jw+1)^2}\]
    Taking fourier transform to find $X(jw)$ from x(t):
    \[x(t)\xleftrightarrow[\text{}]{\text{F.T}}=X(jw)\]
    \[e^{-t}u(t)\xleftrightarrow[\text{}]{\text{F.T}}=\frac{1}{1+jw}\]
    \[Y(jw)=\frac{1}{1+jw}(1-\frac{1}{jw+1}-\frac{1}{(jw+1)^2})\]
    \[Y(jw)=\frac{1}{1+jw}-\frac{1}{(jw+1)^2}-\frac{1}{(jw+1)^3}\]
    Taking the inverse fourier transform to find y(t) from $Y(jw)$:
    \[Y(jw)\xleftrightarrow[\text{}]{\text{F.T}}=y(t)\]
    \[y(t)=(1-t-\frac{t^2}{2})e^{-t}u(t)\]
    \end{enumerate}

\item %write the solution of q2
    \begin{enumerate}
    % Write your solutions in the following items.
    \item Let us find first the frequency response of the system. Assume that: 
    \[y(t) \xleftrightarrow{\mathcal{F}} Y(jw) \hspace{10mm} x(t) \xleftrightarrow{\mathcal{F}} X(jw)\]
    From differentiation property:
    \[\frac{dy(t)}{dt} \xleftrightarrow{\mathcal{F}} jwY(jw)\]
    And from time shift property:
    \[x(t-t_0) \xleftrightarrow{\mathcal{F}} e^{-jwt_0}X(jw)\]
    Then, 
    \[\frac{dy(t)}{dt} = x(t+1) - x(t-1) \xleftrightarrow{\mathcal{F}} jwY(jw) = X(jw)(e^{jw} - e^{-jw})\]
    We find that frequency response is:
    \[H(jw) = \frac{Y(jw)}{X(jw)} = \frac{e^{jw} - e^{-jw}}{jw}\]
    
    \item We can find impulse response of this system by applying inverse transform to frequency response we found in part a. Apply euler's formula first and we get:
    \[H(jw) = \frac{2sinw}{w}\]
    From the table of fourier transform, we get:
    \[h(t) = u(t+1) - u(t-1)\]
    \end{enumerate}

\item %write the solution of q3  
    \begin{enumerate}
    % Write your solutions in the following items.
    \item %write the solution of q3a
    \[x[n]\ast h_1[n] \ast h_2[n]=y[n]\]
    \[x[n]\ast h[n]=y[n]\]
    \[h[n]=h_1[n] \ast h_2[n]  \ \ \ (1)\]
    Since convolution in the time domain is equivalent to multiplication in the frequency domain, we can use fourier transform with $h_1[n]$ and $h_2[n]$ to directly obtain
    $H(e^{jw})$\\
    \[h_1[n] \ast h_2[n]=H_1(e^{jw})H_2(e^{jw}) \ \ \ (2) \]
    We know that;
    \[a^nu[n]\xleftrightarrow[\text{}]{\text{F.T}}=\frac{1}{1-ae^{-jw}}\]
    \[h_1[n] \xleftrightarrow[\text{}]{\text{F.T}} H_1(e^{jw})=\frac{1}{1-\frac{1}{2}e^{-jw}}\]
    \[h_2[n] \xleftrightarrow[\text{}]{\text{F.T}} H_2(e^{jw})=\frac{1}{1-\frac{1}{2}e^{-jw}}\]
    Using eq 1 and 2:
    \[h[n]\xleftrightarrow[\text{}]{\text{F.T}}H(e^{jw})=H_1(e^{jw})H_2(e^{jw})\]
     \[H(e^{jw})=(\frac{1}{1-\frac{1}{2}e^{-jw}})^2\]
    \item %write the solution of q3b
    To find Fourier transform of $x[n]$, let us first find $\mathcal{F}\{sin(\frac{\pi n}{3})\}$ and shift it by $\frac{\pi}{4}$.
    \[\mathcal{F}\{sin(\frac{\pi n}{3})\} = \frac{\pi}{j} \sum_{k = -\infty}^{\infty} \delta(w - \frac{\pi}{3} - 2\pi k) - \delta(w +  \frac{\pi}{3} - 2 \pi k)\]
    \[\mathcal{F}\{sin(\frac{\pi n}{3} + \frac{\pi}{4})\} = e^{jw\frac{\pi}{4}}\frac{\pi}{j} \sum_{k = -\infty}^{\infty} \delta(w - \frac{\pi}{3} - 2\pi k) - \delta(w +  \frac{\pi}{3} - 2 \pi k)\]
    \item %write the solution of q3c
    We know that:
    \[y[n] = x[n] \ast h[n] \xleftrightarrow[\text{}]{\text{$\mathcal{F}$}} H(e^{jw})Y(e^{jw})\]
    Then, we get: 
    \[Y(e^{jw}) = \frac{1}{(1 - \frac{e^{-jw}}{2})^2}e^{jw\frac{\pi}{4}}\frac{\pi}{j} \sum_{k = -\infty}^{\infty} \delta(w - \frac{\pi}{3} - 2\pi k) - \delta(w +  \frac{\pi}{3} - 2 \pi k)\]
    \end{enumerate}

\item %write the solution of q4
    \begin{enumerate}
    % Write your solutions in the following items.
    \item Assume $h[n] = g_1[n] + g_2[n]$ where:
    \[g_1[n] = 2\delta[n] \hspace{10mm} g_2[n] = 2^{-n}u[n]\]
    Using the linearity of discrete Fourier Transformation:
    \[H(e^{jw}) = G_1(e^{jw}) + G_2(e^{jw})\]
    Now, find fourier transformations of $G_1$ and $G_2$ respectively. We know that $\delta[n] \xleftrightarrow[\text{}]{\text{F}} 1$ and $a^n u[n] \xleftrightarrow[\text{}]{\text{F}} \frac{1}{1-ae^{-jw}}$ for $|a| < 1$ from the transform table. Then,
    \[G_1(e^{jw}) = 2\]
    \[G_2(e^{jw}) = \frac{1}{1-\frac{1}{2}e^{-jw}}\]
    \[H(e^{jw}) = 2 + \frac{1}{1-\frac{1}{2}e^{-jw}} \]
    \item We can use the following formula to find the difference equation describing this system:
    \[Y(e^{jw}) = H(e^{jw})X(e^{jw})\]
    Let's arrange frequency response a bit:
    \[H(e^{jw}) = 2 + \frac{1}{1-\frac{1}{2}e^{-jw}} = 2 + \frac{2}{2-e^{-jw}} \]
    Put it into the equation above:
    \[Y(e^{jw}) = (2 + \frac{2}{2-e^{-jw}})X(e^{jw})\]
    \[Y(e^{jw}) = 2X(e^{jw}) + \frac{2X(e^{jw})}{2-e^{-jw}}\]
    \[(2-e^{-jw})Y(e^{jw}) = (4-2e^{-jw})X(e^{jw}) + 2X(e^{jw})\]
    Then use inverse fourier transformation and lookup table. We get:
    \[2y[n] - y[n-1] = 6x[n] - 2x[n-1]\]
    \[y[n] = \frac{y[n-1] + 6x[n] - 2x[n-1]}{2}\]
    \item We can write $x[n] = e^{j\pi n}$ instead of $(-1)^n$ where $w_0 = \pi$. To get $Y(e^{jw})$, we can use the equation below:
    \[Y(e^{jw}) = H(e^{jw})X(e^{jw})\]
    We know $H(e^{jw}) = 2 + \cfrac{2}{2-e^{-jw}}$. Let's find $X(e^{jw})$. From the lookup table:
    \[X(e^{jw}) = 2\pi \sum_{k=-\infty}^{\infty} \delta(w - \pi - 2\pi k)\]
    Put $X(e^{jw})$ into equation.
    \[Y(e^{jw}) = \frac{6-2e^{-jw}}{2-e^{-jw}} 2\pi \sum_{k=-\infty}^{\infty} \delta(w - \pi - 2\pi k)\]
    \end{enumerate}

\end{enumerate}


\end{document}

