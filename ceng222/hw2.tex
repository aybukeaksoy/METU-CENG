\documentclass[12pt]{article}
\usepackage[utf8]{inputenc}
\usepackage{float}
\usepackage{amsmath}
\usepackage{tikz}

\usepackage[hmargin=3cm,vmargin=6.0cm]{geometry}
\topmargin=-2cm
\addtolength{\textheight}{6.5cm}
\addtolength{\textwidth}{2.0cm}
\setlength{\oddsidemargin}{0.0cm}
\setlength{\evensidemargin}{0.0cm}
\usepackage{indentfirst}
\usepackage{amsfonts}

\begin{document}

\section*{Student Information}

Name : Aybüke Aksoy\\

ID : 2448090\\


\section*{Answer 1}
\paragraph{a)}
If X and Y are independent, the joint density function f(x,y) must be equal to f(x)f(y).\\
We found f(x) and f(y) in part b. 
\[f_X(x)f_Y(y)=\frac{2\sqrt{1-x^2}}{\pi}\frac{2\sqrt{1-y^2}}{\pi} \neq \frac{1}{\pi} \ , \ (x^2+y^2\leq1)\]
Since they are not equal, X and Y are not independent.

\paragraph{b)}
We can calculate the marginal pdfs from:
\[f_X(x)=\int f_{X,Y}(x,y)dy\]
\[f_Y(y)=\int f_{X,Y}(x,y)dx\]
$f_{X,Y}(x,y)=0$ when x and y does not satisfy the equation $x^2+y^2 \leq 1$. Therefore, integrals are zero at those points and we do not have to calculate it.\\
At other points, we can use the equation $x^2+y^2 = 1$ to find the lower and upper limit of the integral for x and y. 
\[y_1=\sqrt{1-x^2} \ , \ y_2=-\sqrt{1-x^2}\]
\[x_1=\sqrt{1-y^2} \ , \ x_2=-\sqrt{1-y^2}\]
\[f_X(x)=\int_{-\sqrt{1-x^2}}^{\sqrt{1-x^2}}\frac{1}{\pi}dy = \frac{2\sqrt{1-x^2}}{\pi}\]
\[f_Y(y)=\int_{-\sqrt{1-y^2}}^{\sqrt{1-y^2}}\frac{1}{\pi}dx = \frac{2\sqrt{1-y^2}}{\pi}\]
\paragraph{c)}
We know that:
\[E(X)=\int xf(x)dx\]
Also we know f(x) from part b, so we can calculate E(X) as:
\[E(X)=\int_{-1}^{1}\frac{x2\sqrt{1-x^2}}{\pi}dx\]
By change of variable $u=1-x^2$,
du=-2xdx and both the lower and upper limit of the integral becomes 0.
\[E(X)=\int_{0}^{0}\frac{-\sqrt{u}}{\pi}du\]
Hence, 
\[E(X)=0\]
\paragraph{d)}
We know that:
\[Var(X)=\int x^2f(x)dx-\mu^2\]
Taking the integral by again using the f(x) we found from part b:
\[Var(X)=\int_{-1}^{1}\frac{x^22\sqrt{1-x^2}}{\pi}dx=0.25\] 
And we already know the value of $\mu$ from part c as it is equal to E(X).
\[Var(X)=0.25-0^2=0.25\]
\section*{Answer 2}
\paragraph{a)}
Joint density function for a uniform distribution is 
\[f_x(x)=\frac{1}{b-a}\]
For $T_A$ and $T_B$ b=100 and a=0;
\[f_{T_A}(t_a)=\frac{1}{100} \ , \ f_{T_B}(t_b)=\frac{1}{100} \ , \ 0<t_a,t_b<100\]
Since they are independent, 
\[f_{T_A,T_B}(t_a,t_b)=f_{T_A}(t_a)f_{T_B}(t_b)=\frac{1}{100}\frac{1}{100}=\frac{1}{10000} \ , \ 0<t_a,t_b<100\] 
Joint CDF can be found from the density functions as:
\[F_x(x)=\int f_x(x)dx\]
\[F_{T_A}(t_a)=\int_{0}^{t_a}f(t_a)dt_a=\frac{t_a}{100} \ , \ F_{T_B}(t_b)=\int_{0}^{t_b}f(t_b)dt_b= \frac{t_b}{100} \ , \ 0<t_a,t_b<100\]
Since they are independent, 
\[F_{T_A,T_B}(t_a,t_b)=F_{T_A}(t_a)F_{T_B}(t_b)=\frac{t_a}{100}\frac{t_b}{100}=\frac{t_at_b}{10000} \ , \ 0<t_a,t_b<100\] 
\paragraph{b)}
We can find the probability that subject A pushes the button in the first 10 seconds and subject B pushes the button in the last 10 seconds bu using the Joint CDF's we found in part a:
\[P\{T_A \leq 10\}=F_{t_a}(10)=\int_{0}^{10}\frac{1}{100}dt_a=\frac{1}{10}\]
\[P\{T_B \geq 90\}=1-F_{t_b}(90)=1-\int_{0}^{10}\frac{1}{100}dt_b=1-\frac{9}{10}=\frac{1}{10}\]
Since these events are independent, 
\[P\{T_A \leq 10 , T_B\geq 90\}=P\{T_A \leq 10\}P\{T_B \geq 90\}=\frac{1}{100}\]
\paragraph{c)}
Let $t_b$ be the second B pushes the button and $t_a$ be the second A pushes the button.\\
Since subject A can push the button at most 20 seconds after subject B, unless B pushes the button at $t_b > 80$ ,  $t_a$ can only get values between 0 and $t_b+20$. When $t_b > 80$, $t_a$ can get the values from 0 to 100 as the difference between $t_a$ and $t_b$ is always less than 20. 
\begin{center}
        \begin{figure}[H]
        \hspace{40mm}\includegraphics[scale=0.5]{pic2.png}
        \caption{Graph of $t_b$ vs $t_a$}
        \label{fig:htgraph}
    \end{figure}
\end{center}
The ratio of the area under the curve to the total area which is 100x100 defines the probability that subject A pushes the button at most 20 seconds after subject B. \\
Taking integral to find the probability:
\[\frac{1}{10000}\int_{0}^{80}(t_b+20)dt_b \ + \  \frac{1}{10000}\int_{80}^{100}(100)dt_b\]
\[0.48+0.20=0.68\]
\paragraph{d)}
Let $t_b$ be the second B pushes the button and $t_a$ be the second A pushes the button.\\
Since the elapsed time of subject B and A cannot differ by more than 30 seconds, when B pushes the button at $0<t_b<30$ $t_a$ can take values between 0 to $t_b+30$. When $30<t_b<70$, $t_a$ can take values between $t_b-30$ to $t_b+30$. When $70<t_b<100$, $t_a$ can take values between $t_b-30$ to $100$.
\begin{center}
        \begin{figure}[H]
        \hspace{40mm}\includegraphics[scale=0.5]{pic1.png}
        \caption{Graph of $t_b$ vs $t_a$}
        \label{fig:htgraph}
    \end{figure}
\end{center}
The ratio of the area between the two curves to the total area which is 100x100 defines the probability that they pass the test. \\
\[\frac{1}{10000}\int_{0}^{30}(t_b+30)dt_b \ + \  \frac{1}{10000}\int_{30}^{70}(t_b+30)-(t_b-30)dt_b \ + \ \frac{1}{10000}\int_{70}^{100}100-(t_b-30)dt_b \]
\[=0.135+0.240+0.135=0.51\]
\section*{Answer 3}
\paragraph{a)}
\[F_{X_i}=P(X_i \geq x)=e^{-\lambda_ix_i}\]
To find the $F_{X_n}=P(T \leq t)$:
\[P(T \leq t)=1-P(T \geq t)\]
\[=1-P(min\{X_1,X_2,X_3....X_n \} \geq t)\]
Since T is the minimum which means it is less than all $X_i$'s and greater than t, all $X_i$'s will be greater than t. 
\[=1-P(X_1\geq t,X_2 \geq t,X_3 \geq t,....X_n \geq t)\]
Since those are independent,
\[=1-P(X_1 \geq t)P(X_2 \geq t)P(X_3 \geq t).....P(X_n \geq t)\]
\[=1-e^{-\lambda_1 t}e^{-\lambda_2 t}e^{-\lambda_3 t}.....e^{-\lambda_n t}\]
\[=1-e^{-\lambda_1t-\lambda_2t-\lambda_3t.....-\lambda_n t}\]
\[=1-e^{-\Sigma_{i=1}^{n}\lambda_i t}\]

\paragraph{b)}
$\mu$ is given as $\frac{10}{n}$ in the question.\\
Since $\mu=E(X)=\frac{1}{\lambda}$, we know that:
\[\lambda_n=\frac{n}{10}\]
\[\lambda_1=\frac{1}{10} \ ,\ \lambda_1=\frac{1}{10} \ ,\ \lambda_2=\frac{2}{10}......\lambda_10=\frac{10}{10} \]
\[\Sigma_{n=0}^{n=10}\lambda_n=\frac{55}{10}\]
Exponential distribution has the density function:
\[f(t)=\lambda e^{-\lambda t}\]
\[f(t)=\frac{55}{10}e^{-\frac{55}{10} t}\]
Finding the expected time:
\[E(X)=\int_{0}^{\propto}tf(t)dt\]
\[E(X)=\int_{0}^{\propto}tf(t)dt\]
\[E(X)=\int_{0}^{\propto}t\frac{55}{10}e^{-\frac{55}{10} t}dt\]
\[=0.\overline{18}\]

\section*{Answer 4}
We are gonna apply the central limit theorem and use continuity correction. 
\[\mu=np \ , \ \sigma=\sqrt{np(1-p)}\]
\subsection*{a)}
Probability of being an undergraduate student is 0.74, p=0.74
The number of students participating in a poll is 100, n=100
\[\mu=np=74 \ , \ \sigma=\sqrt{np(1-p)}=\sqrt{74(1-0.74)}=\sqrt{19.24}\]
Let U denote the random variable of participants being an undergraduate students. \\
Instead of calculating the probability that at least $\%70$ of the participants being undergraduate students, we can calculate the probability that at most $\%70$ (without including 70) of the participants being undergraduate students and substract it from 1. 
\[P(U < 70)=P(U<69.5)=P \Big{\{}\frac{U-\mu}{\sigma}<\frac{69.5-\mu}{\sigma}\Big{\}}\]
\[\approx P \Big{\{}\frac{69.5-74}{\sqrt{19.24}}\Big{\}}=\Phi(-1.026)\]
Hence, 
\[P(U \geq 70) \approx 1-\Phi(-1.026)=0.8485\]
\subsection*{b)}
Probability of pursuing a doctoral degree is 0.1, p=0.1
The number of students participating in a poll is 100, n=100
\[\mu=np=10 \ , \ \sigma=\sqrt{np(1-p)}=\sqrt{10(1-0.1)}=\sqrt{9}=3\]
Let D denote the random variable of participants pursuing a doctoral degree.
\[P(D \leq 5)=P(D < 5.5)=P \Big{\{}\frac{D-\mu}{\sigma}<\frac{5.5-\mu}{\sigma}\Big{\}}\]
\[\approx P \Big{\{}\frac{5.5-10}{3}\Big{\}}=\Phi(-1.5)\]
Hence,
\[P(D \leq 5) \approx \Phi(-1.5)=0.066807\]
\end{document}

